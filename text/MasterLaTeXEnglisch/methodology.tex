\chapter{Methodology}\label{Methodology}
\section{motor learning model}
This work optimizes a motor spicing model introduced by Spüler et al. \cite{sebastianPaper} which is an extension of the work of Chadderdon et al. \cite{chadderdonNeuronalModel}. The model presents a biologically realistic model of the motor cortex, which uses reinforcement learning to reach a target angel with a simulated forearm.
As a standard reinforcement model \cite{reinforcementlearning}\cite{chadderdonNeuronalModel} it consists of an actor, that maps perceptions to actions a environment that reacts and a critic providing reward or punishment feedback to the actor. Spüler et al. and Chadderton et al. realized the environment as a forearm model with a one degree of freedom joint. Therefore the forearm could be moved either up or down. The joint angle limited by 0\degree and 135\degree. Whereby 0\degree means fully straightened and 135\degree means fully flexed. The actor got implemented with a spiking neural network. A spiking neural network is a weighted connected graph with some input and output nodes. Each edge is called a connection and each vertex called a neuron. The network calculates a time depending mapping from some input to output values. %TODO cite! oder ausführlicher erklären. ANN and SNN. Oder al bekannt annehmen...
%vielleicht hier erst grobe struktur erklären, dann neurons im detail.
Each neuron in this model represents an either excitatory or inhibitory dynamic unit. Excitatory neurons excite connected neurons, while inhibitory neurons inhibit connected neurons.
The dynamics or in other words the time depending state of each neuron is represented by the membrane potential $V_m(t)$.$V_m(t)$ is modelled by an differential equation introduced by Izhikevich \cite{izhikevichSimpleModel}, which is a good and efficient approximation of the Hodgkin-Huxley model \cite{hodgkinHuxleyModel}. Izhikevich's model slightly adapted by Spüler et al. with noise is described as follows (variable names changed):
\begin{align*}
	V_m(t)' &= 0.04 V_m(t)^2+5V_m(t)+140-u+I(t)+V_n(t)\\
	u(t)' &= a(b\cdot V_m(t)-u(t))\\
	&\text{reset after spike:}\\
	\text{if } v&\geq V_t,\text{ then}
	\begin{cases}
	1 & \leftarrow V_r\\
	0 & \leftarrow u+d\\
	\end{cases} 
\end{align*}
where $u$ represents the membrane recovery variable. $I$ defines the synaptic input currents. That is the sum of weighted inputs to a neuron.(TODO ref to ANN fundamentals) The firing pattern of a neuron depend on the choice of parameters $a,b,d,V_r$ and $V_t$. Different resulting spiking patterns can be seen in appendix TODO  refpicture. $V_r$ is the resting potential of the neuron. $V_t$ the spiking threshold. $V_n$ is a 300Hz noise input that leads to motor babbling. That is, network send output signals although no input signal is given. This behaviour is important for reinforcement learning since if the actor doesn't produce any actions the critic isn't able to give feedback \cite{chadderdonNeuronalModel}.The parameter a,b,c,d are chosen as one can shown in table \ref{table:DynamiModelParams}.
%allgemein fehlt oft der Bezug auf den Motor cortex und ob das dort ähnlich  ist
%TODO should be past tense because was done and not universal valid 
\begin{table}[h]
	\centering
	\begin{tabular}{ |c||c|c|  }		
		\hline
		Parameter & Excitatory & Inhibitory \\
 		\hline
 		$a$&$0.02$& $0.02+0.08r_i$\\
 		$b$&$0.2$& $0.25-0.05r_i$\\
 		$d$&$8-6\cdot r_i^2$& $2$\\
 		$V_r$&$-65$& $-63$\\
 		\hline
	\end{tabular}
	\caption[Dynamic model's parameters ]{Parameters used in the dynamic model }
	\label{table:DynamiModelParams}
\end{table}

%Noice:
% how noice is calculated not important

$r_i$ is uniform distributed in  $[0,1]$. For excitatory neurons moving $r_i$ from $0$ to $1$ leads to a transition from a regular to a chattering spiking pattern. For inhibitory neurons from a fast to a low-threshold spiking pattern. Examples of those patterns can be seen in (TODO ref appendix.).
The neurons are further divided into 5 logical groups: proprioceptive (P), excitatory sensory (ES), inhibitory sensory(IS), excitatory motory (EM) and excitatory sensory (IS) neurons.  Spüler's et al. model contains 48 P, 96 ES, 32 IS, 48 EM and 32 IM cells. Which leads to a total amount of 256 neurons.
Each logical group is connected with a specific probability as shown in \ref{table:SpuelerConnectionProbs}:

\begin{table}[h]
	\centering
		\begin{tabular}{ |c|c|c|c|c|c|  }
			\hline
			   & P &  EM  & IM   & ES   & IS    \\ \hline
			P  & 0 &  0   & 0    & 0,1  & 0      \\
			EM & 0 &  0   & 0,43 & 0    & 0     \\
			IM & 0 & 0,44 & 0,62 & 0    & 0      \\
			ES & 0 & 0,08 & 0    & 0    & 0,43   \\
			IS & 0 &  0   & 0    & 0,44 &  0,62 \\ \hline
		\end{tabular}
	\caption[Spüler's et al. model's connection probabilities ]{Connection probabilities of neuron types in the model of Spüler's et al.}
	\label{table:SpuelerConnectionProbs}
\end{table}
%firing rates:
% not important
%
Based on these connections the resulting network is shown in figure (TODO figure network)


% TODO structure: RF learning, Arm, neuro model. Dynamics, movement coding, learning

The motor control cycle works as follows: A new movement angle is encoded by the EM neurons, after a time gap of 50 ms the forearm moves. At the same time reward or punishment is send. After a second time step of 25 ms the new position angle is encoded by the proprioceptive cells. \cite{sebastianPaper} grounds the time gaps on peripheral and sub cortical processing delays. Why there is an additional delay for the P cells to fire but not for the reward keeps ungrounded. The whole system works with a frequents of 0.02kHz. That means the EM neurons encode a new angle every 50ms. An overview of this process can be seen in figure ref TODO.
The time needed by EM cells to encode a new angle is again 50ms. EM cells are parted into two equal sized groups of 24 neurons. The first 24 are in the first group the next 24 in the second. Each firing of a cell in the first group leads to an 1\degree  downward motion of the arm; in the lower part it leads to an 1\degree  upward motion. Over the time window of 50ms all spikes in each group are summed up. The resulting movement angle the difference of those sums.
Whether reward or punishment is send by the critic is decided whether the distance $\Delta\theta_t $of the current angle $\theta_t$  to the target angle $\theta_{target}$ got smaller or higher compared to the mean of the last two differences $\Delta\theta_{prev}$. This can be seen in the following formula:
\begin{align*}
	\Delta\theta_t &= |\theta_t-\theta_{target}|\\
	\Delta\theta_{prev} &= \left| \dfrac{\theta_{t-1}+\theta_{t-2}}{2} -\theta_{target}  \right |\\
	\text{critic response} &= 
	\begin{cases}
	\text{reward} &  \quad \text{if } \Delta\theta_t < \Delta\theta_{prev}\\
	\text{punishment}& \quad \text{if } \Delta\theta_t > \Delta\theta_{prev}\\
	\text{no response}& \quad \text{if } \Delta\theta_t = \Delta\theta_{prev}\\
	\end{cases}
\end{align*}
Note that reward or punishment doesn't influence all connections. Connections according to the Hebbian theory \cite{originalHebbianLaw} are only able to learn if a post-synaptic spike followed a pre-synaptic spike in the 50 ms where the movement to the current angle got encoded. Further on only connections from ES to EM neurons are able to learn. (TODO Why find prove ).  Reward or punishment increases or decreases a weight scale factor of each connection that is able to learn. For more informations look Appendix. TODO just copy picture.
A weight lies in the interval [0,5].

TODO population coding of P cells.
TODO picture structure
TODO Own plot time.
TODO second Model
TODO Appendix picture weight formula
TODO fireing patters.
