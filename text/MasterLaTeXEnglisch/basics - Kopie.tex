%%%%%%%%%%%%%%%%%%%%%%%%%%%%%%%%%%%%%%%%%%%%%%%%%%%%%%%%%%%%%%%%%%%%
% Basic Informations
%%%%%%%%%%%%%%%%%%%%%%%%%%%%%%%%%%%%%%%%%%%%%%%%%%%%%%%%%%%%%%%%%%%%

\chapter{Basics}\label{Basics}
% todo explain EA, Gene, Genome, Allel. 


\section{NEAT}\label{NEAT} % todo paper reference
\Gls{neat} (\cite{NEAT}) is an neuroevolution method that genetically encodes and evolves weights and topologies of neural networks. \gls{neat} stands out from other neuroevolution algorithms by meaningful crossovers in disparate ANN topologies, protection of topological innovations and a efficient search by a stepwise increasing search space. (TODO cite some). Furthermore \gls{neat} is well suited for reinforcement learning problems. 
\subsection{biological basics} 
% maybe definitions are better ask %MKK zitieren %% todo fachwerk aus bib
The whole inheritable information needed to create a new organism is called genome. A genome consists of genes which are responsible for the expression of specific features. Genes that express same features with  different characteristics are called alleles. Assuming the eye colour is expressed by one gene.  Variations of this gene which lead to different eye colours are alleles. (TODO find proven example with one gene)
\subsection{Evolutionary algorithm} % Todo abkürzung EA
Evolutionary algorithms (EAs) (see generally \cite{book:IntroductionEA}) are algorithms that mimic the natural evolution process. That is that only the fittest organisms of a generation will survive. An EA maintains a group of solutions called a population. A solution in a population is called an individual. Each individual has a numerical fitness value that determines how good this solution is. An individual is represented by a code also called genome. During the run of an EA a genome of an individual undergoes several variation operations. One typical operations are mutations, where parts of the genome are changed randomly. Cross Over as an other typical operation exchanges parts of the genome of two individuals. Usually an EA performs those operations to generations and only the fittest childs are s taken over into a new generation. The function giving the fitness of an individual is called objective function. EAs are mainly used to find good local maxima of objective functions that are either unknown or not analytical differentiable and not solvable with numerical methods like gradient descends.  (TODO nachlesen vergleichn script zell)


\subsection{operating principle} 
The genetic basic of \gls{neat} is a direct encoded genome, which contains connection and node genes. Each node gene evolves a neuron while each connection gene represents a weighted and directed connection between two neurons. A node gene contains it's identification number and it's type. Possible types are Input, Output and Hidden. A connection gene encodes its connection by referring to two node gene identification numbers. In addition a connection gene contains its weight, an expression flag and an innovation number. The expression flag indicates whether the expression gets realised or not. The innovation number is the fixed identification number of a connection gene. Alleles of a connection gene share the same node genes and innovation number but can have different weight and expression flag values. Every time a new neuron gene or connection gene is created a global counter number is increased by one and than used as the new innovation number. In the case that the same connection gets created in two different offsprings of a generation they get the same innovation number assigned. This only applies if it happens in the same generation. Thus one gene can only evolve in one particular generation. As a result the innovation number assures that connection genes with the same innovation number connect the same nodes but not all connection genes connecting the same nodes have the same connection number.(this represents the biological fact, that the same feature can be expressed by different genes.)
Three types of mutation are realized. A \textit{w